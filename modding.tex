\subsection{Modding}
Modding consists in altering an existent video game either by completely changing its gameplay experience or by changing only its visual aspect.
Some mods add new character, items, enemies, altered weapons while others only change the way the game looks, either by changing the \ac{HUD}\footnote{Heads Up Display - is a game interface used to display information to the players while in-game} or the visual effects of certain actions or items.

However there are mods that go even further by adding complete new story lines or even creating new experiences in the original games.
One of the most successful examples is the \ac{CS} mod, which completely changed Half-Life, a first person shooter action game by Valve Software.
\ac{CS} had such a success that millions of players preferred to play the mod over the original game.
Upon realising this trend, Valve Software changed it's game development and business model policies in order to include game modding as part of the gameplay experience \cite{scacchi:mods}.
More recently \ac{CS} was even released as a stand-alone game in Valve Software's popular game distributing platform, Steam.

Despite Valve Software's embrace of modding, different companies have different views on modding.
For instance, Blizzard Entertainment, publisher of World of Warcraft, one of the most played MMORPG\footnote{Massively Multi-player Online Role Playing Game} of all time; does not allow modders to change the in-game mechanics of the game, being only allowed to change the game's \ac{HUD}.

Many companies are, in certain ways, encouraging players to modify and improve their games, either by providing tools for that purpose or by providing extensive documentation.
Bethesda Softworks, Mojang, Crytek, Firaxis, Valve Software, Klei Entertainment, are some examples of such companies.
Some of them, like Klei Entertainment, maintain forums and have staff dedicated to accompany and help mod makers with their ideas and problems.

As a community driven activity (mods are made by the community, for the community), they are usually distributed through the Internet.
Several platforms provide means of distribution, being three of the most popular NexusMods\footnote{http://www.nexusmods.com}, ModDB\footnote{http://www.moddb.com}, and Steam Workshop\footnote{https://steamcommunity.com/workshop/}.
Through these platforms players can download and install mods for their favourite games or share their own creations with other players.

Survival games have followed this trend of allowing users to modify the game's content.
Notably, \ac{DST} from Klei Entertainment, has made modding a living part of the game play experience.
By subscribing to a mod in Steam's Workshop, players can experience new game play mechanics, play with different characters, craft new weapons and items, visit new in-game areas, etc., Klei Entertainment puts no barrier on what mods can do (the original game content is a mod itself).

