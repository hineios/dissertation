% #############################################################################
% RESUMO em Português
% !TEX root = ../main.tex
% #############################################################################
% use \noindent in first paragraph
\noindent Os videojogos de hoje em dia batalham por manter altos niveis de fidelidade.
As representações altamente realistas dos mundos virtuais apenas são traídas pela falta de credibilidade dos personagens que os habitam.
De modo a manter a imersão criada por gráficos extraordinários, os personagens têm de ser capazes de criar a ilusão de vida, o que requer que possuam características humanas como consciência social, reactividade e procura activa de objectivos.
Alguns géneros de jogos, como RPGs, já viram este problema ser endereçado pela utilização de modelos de agência originários em grupos académicos.
No entanto, videojogos de sobrevivência ainda não foram alvo deste tipo de atenção.
Neste trabalho, endereçou-se esta falha propondo um sistema que permite criar personagens baseados em modelos de agência para videojogos de sobrevivência.
Usando o FAtiMA Toolkit, um modelo de agência por direito, e o Don't Starve Together, um videojogo de sobrevivência popular, implementou-se e publicou-se o sistema proposto e um personagem de exemplo, o Walter.
Adicionalmente, testou-se e comparou-se o Walter com um personagem baseado em arvores de comportamento.
