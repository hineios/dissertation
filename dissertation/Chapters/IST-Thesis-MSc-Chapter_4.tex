% #############################################################################
% This is Chapter 4
% !TEX root = ../main.tex
% #############################################################################
% Change the Name of the Chapter i the following line
\fancychapter{Conceptual Model}
\cleardoublepage
% The following line allows to ref this chapter
\label{chapter:conceptual-model}


\noindent In this chapter we present a conceptual representation of our work.
Throughout this chapter we will describe each of the components of the model and its expected behaviour.
By the end of this chapter, the transition from the model, presented here, to the implementation, presented in Chapter \ref{chapter:implementation}, should be clear. 

In Figure \ref{fig:model} we can see several different modules and concepts.
The presented decoupling between the agent's \ac{AI} and the agent's embodiment, allows us to consider the use of different \ac{AI} techniques and different survival games.

Do notice that, despite the fact that the agent's \ac{AI} is represented outside the visual representation of the survival game, it needs not be the case.
Both concepts can be as tightly or as loosely coupled as needed for each different implementation.

Nonetheless, the Survival Game will need to provide ways for an \ac{NPC} to interact with the world, both by collecting information of the world using sensors and by making changes to the world through the use of actuators.
The \ac{NPC} will then extract the necessary perceptions about the world and, via a specified channel of communication, send them to the appropriate \ac{AI}.
The \ac{AI} will then perceive the received perceptions.
Whenever necessary, the \ac{AI} must emit actions that will be executed by the \ac{NPC} via the use of its actuators.
Throughout this execution, a Logger component will register all the executed actions, their outcome, and all the perceptions sent to the \ac{AI}.

\begin{figure}
  \centering
  \includegraphics[width=\textwidth]{./Images/model}
  \caption{The proposed conceptual model.}
  \label{fig:model}
\end{figure}

Now that we've explored the general flow of the model, we'll describe each of its components in detail.

\begin{description}
	\item[\textbf{AI Agents}] \hfill
    
    This represents the \ac{AI} that will effectively control an \ac{NPC}.
    It is responsible for receiving perceptions and make decisions.
    It must be able to handle behaviour decisions and non-behaviour related decisions (such as dialogue).
    
    This module will need to have reactive capabilities as well as strategic decision making.
    Due to their dangerous environments and sometimes monsters, survival games (and in specific Don't Starve Together) require the ability to handle unexpected events, e.g. when a monster comes to attack the character.
    Additionally, they also require strategic decision making in order to better survive the world.
    The harshness of seasons will require that characters prepare ahead to better survive.
    
    Additionally, the capability for dialogue should be considered.
    Great part of player to \ac{NPC} interaction is based on dialogue, as such, this module should be able to handle dialogues.
    
    \item[\textbf{Dispatcher}] \hfill
    
    This module represents a bridge between the \ac{AI} and the embodiments of agents, managing the creation and destruction of the necessary \ac{AI} modules in accordance with the existing embodiments (\acp{NPC}).
    It is responsible to direct each perception and action to the appropriate recipient, be it the \ac{AI} or the embodiment of the agent.
    
    \item[\textbf{Communication Protocol}] \hfill
    
    Only present for illustrative purposes, the communication protocol is the bridge where perceptions and actions travel back and fort.
    This can be a simple local method invocation or a more complex communication system such as a remote method invocation over the network.
    
    Preferably the former, which presents significant advantages: reduced response time, no worries about availability of external services, and no need to translate information between potentially different systems.
    But, in the case of the later, it must handle the translation of information between systems transparently.
    
    \item[\textbf{World}] \hfill
    
    The virtual world of the game.
    This is where all the characters (\acp{NPC} and player characters) actually live and must provide ways for the \ac{NPC}s to act upon it and perceive it.

    \item[\textbf{Scenario Configurator}] \hfill
    
    This is, ideally, a part of the survival game.
    It provides a simple set of configurations that a developer can specify to configure the scenario that will be executed.
    It can also incorporate some configurations for the \ac{AI} that will empower the \acp{NPC}.
    
    \item[\textbf{NPCs}] \hfill
    
    This module represents the embodiment of the agents.
    It will execute any actions on the world and collect the necessary perceptions which in turn are sent to the \ac{AI}.
    
    \item[\textbf{Logger}] \hfill
    
    The Logger will be used to keep a registry of all the executed actions and their outcomes.
    It will also keep the registered perceptions of the \ac{NPC}.
    Although its representation puts it as a part of the game, this module can be independent from the game itself.
    
    \item[\textbf{User Interface and Players}]
    
    These modules are illustrative of the presence of players characters in the world.
    The survival game must accommodate the presence of several players that can play cooperatively or competitively and communicate with each other and the \acp{NPC}.
    
\end{description}

