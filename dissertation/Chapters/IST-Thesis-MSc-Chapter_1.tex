% #############################################################################
% This is Chapter 1
% !TEX root = ../main.tex
% #############################################################################
% Change the Name of the Chapter i the following line
\fancychapter{Introduction}
\cleardoublepage
% The following line allows to ref this chapter
\label{chapter:introduction}

\noindent In recent years, the continuous improvement and development in computer graphics has pushed video games to new levels of graphical fidelity.
Graphical representations of virtual worlds have become increasingly more realistic, allowing players to experience new levels of immersion \cite{brown:immersion}.
Additionally, with the recent boom in virtual reality systems, players have never been so physically immersed in a game's virtual world.

The increases on fidelity and lifelikeness of the virtual world cause players to create expectations on the interactions they can have on the world.
This expectation applies not only to interactions with the virtual world itself, but is also extended to the interactions with the characters that compose the world, typically called \acp{NPC}.

Computer and player controlled characters need to act in a believable manner so that the illusion of reality created by exquisite graphics and physics, the player's immersion, is not broken \cite{thrainsson:emotion-games}.
The lack of believability in \acp{NPC} can break the player's immersion and can provide a bad gameplay experience, given that the game's flow and player's immersion are key elements to the gameplay experience \cite{ijsselsteijn:userexperience}.
Reactivity, goals, emotions, and social competence are necessary traits \acp{NPC} need in order to be believable \cite{bates:emotioninagents}.

Some game genres, like \acp{RPG}, which are heavily dependent on player to \ac{NPC} interaction, have received special attention on the creation of believable characters.
However, other game genres, like survival games, have not received such attention.

Survival games, a genre that has seen a rise in popularity recently, are often based in open world scenarios that can be procedurally generated.
Either single or multi-player, some survival games revolve around resource collecting, crafting, and management of the character's needs (hunger and thirst, mostly), while others are set in supernatural worlds with some elements of horror: monster infested worlds where players have to survive by defeating them.

%Their popularity can be justified for their ability in making experiences like surviving a night or overcoming a simple enemy fun.
More often than not, survival games lack a specific objective other than surviving in a harsh world for as long as possible.
However, survival games are trending among players and game developers alike, seeing great attention in the indie game scene.

Don't Starve \cite{games:dontstarve}, for example, is a single player survival game set on a procedurally generated world where players have to collect, craft, explore, and fight to survive.
With a day and night cycle, Don't Starve provides an interesting gameplay by combining mechanics for temperature, wetness, hunger, health, and sanity.
The introduction of sanity to the game's character provides a novelty factor among its peers, giving the character a human-like trait appropriate for a survival game, representing the strain put on the character by having to survive the harshness of a world with nothing to start with.

Following the trend for multiplayer survival games, Don't Starve has also been released as a multiplayer game under the name of \ac{DST} \cite{games:dontstarvetogether}.
The game is identical to its predecessor, though it allows several players to join the same world.
The multiplayer nature of the game brings new challenges to players that can either cooperate or compete for survival.
In extreme cases, players can even attack and kill each other giving the game an extra challenge to overcome.

However, the game counts with no \acp{NPC} to accompany players in their journey for survival, making players dependent on each other for gameplay.
The few autonomous non-aggressive characters of Don't Starve Together present limited behaviour and are not subject to the same rules as the players (hunger, health, sanity, etc.).

The nonexistence of interesting and robust \ac{AI} for characters is not exclusive to Don't Starve Together.
Other survival games suffer from the same problem: the lack of collaborative \ac{AI} for \acp{NPC} capable of creating believable characters.

As said before, on other game genres like \acp{RPG}, tools exist and have been used for the creation of \ac{AI} controlled \acp{NPC} that take into account the character's believability \cite{guimaraes:cif-ck}\cite{afonso:agents-that-relate}\cite{ferreira:merchant-model}.
Usually empowered by social models, these tools have not yet been applied to survival games.

\section{Motivation}

\noindent In order to improve player's game play experience, the game industry has used \ac{AI} with several different purposes: player experience modeling, procedural content generation, massive-scale game data mining, and \ac{NPC} \ac{AI} \cite{yannakakis:gameairevisited}.

Many modern day video games are dependent on player to \ac{NPC} interaction, and while
some degree of independent decision making is implemented through the use of \ac{AI} techniques, it’s mostly based on combat and has no social concern whatsoever.
This lack of social ability in \acp{NPC} can badly impact their believability and in turn affect the player's gaming experience.

While some work has been done to tackle the problem of believability in \acp{NPC}, survival games however, have not been subject of this effort.

\section{Objectives}

\noindent As our objective for this work, we propose the development of a platform that will enable the creation of \acp{NPC} controlled by agency based models for survival games. 
By making use of \ac{FAtiMA}, an agency based model, and \ac{DST}, a survival game, we will create a framework where agents can be created with FAtiMA and played in Don't Starve Together.

\section{Contribution}

\noindent With the conclusion of this project we've successfully developed the following assets:
\begin{description}
  \item[FAtiMA-DST] \hfill \\ A modification for \ac{DST} that enables us to run FAtiMA controlled characters inside the game.
  \item[FAtiMA-Server] \hfill \\ A console application the handles all the FAtiMA related processing for controlling characters.
  \item[Walter - The AI Companion] \hfill \\ An example \ac{NPC} published in Steam Workshop that makes use of FAtiMA-DST and FAtiMA-Server.
  \item[Tutorials \& Instructions] \hfill \\ A complete set of instructions publicly available in conjunction with FAtiMA-Server and FAtiMA-DST that enable developers to create their own FAtiMA powered \acp{NPC} for \ac{DST}.
\end{description}

\section{Outline}

\noindent In Chapter \ref{chapter:related-work} we will begin by exploring how the game industry is currently developing \acp{NPC}, then we'll explore some of the existing tools and frameworks for building agents, and terminate the chapter with the presentation of some agency models.

Then, in Chapter \ref{chapter:case-study-dst} we will focus on the game itself providing a comprehensive view of it's core mechanics, challenges, and describe a general play-through of the game.
The conceptual model of the framework is presented in Chapter \ref{chapter:conceptual-model}, while Chapter \ref{chapter:implementation} depicts all the implementation details as well as a walk-through of the process of creating an \ac{NPC} for \ac{DST}, using the example character, Walter, as a base.

The results of this work are presented in Chapter \ref{chapter:evaluation}. In this chapter we'll compare our example \ac{NPC}, Walter, with an unpublished work from an anonymous \textit{modder} and present a use case of our platform.
Finally, in Chapter \ref{chapter:conclusion}, the conclusions of this work are presented as well as some remarks on possible future work.